\subchapter{Filesystems - Block file systems}{Objective: configure and
  boot an embedded Linux system relying on block storage}

After this lab, you will be able to:
\begin{itemize}
\item Produce file system images.
\item Configure the kernel to use these file systems
\item Use the tmpfs file system to store temporary files
\item Load the kernel and DTB from a FAT partition
\end{itemize}

\section{Goals}

After doing the {\em A tiny embedded system} lab, we are going to copy
the filesystem contents to the SD card. The storage will be
split into several partitions, and your board will boot
on an root filesystem on this SD card, without using NFS anymore.

\section{Setup}

Throughout this lab, we will continue to use the root filesystem we
have created in the \code{$HOME/__SESSION_NAME__-labs/tinysystem/nfsroot}
directory, which we will progressively adapt to use block filesystems.

\section{Filesystem support in the kernel}

Recompile your kernel with support for SquashFS and ext4\footnote{Basic
configuration options for these filesystems will be sufficient. No need
for things like extended attributes.}.

Update your kernel image in the boot partition.

Boot your board with this new kernel and on the NFS filesystem you
used in this previous lab.

Now, check the contents of \code{/proc/filesystems}. You should see
 that ext4 and SquashFS are now supported.

\section{Add partitions to the SD card}

Plug the SD card in your workstation.

Using \code{parted /dev/mmcblk0}, add two partitions, starting from the beginning
of the remaining space, with the following properties:
\begin{itemize}

\item One partition, for the root filesystem, 8 MB big:
\if\defstring{\labboard}{stm32mp1}
  \begin{verbatim}
  mkpart rootfs 131072s 147455s
  \end{verbatim}
\fi
\if\defstring{\labboard}{stm32mp2}
  \begin{verbatim}
  mkpart rootfs 129538s 145921s
  \end{verbatim}
\fi

\item One partition, that fills the rest of the SD card, that will be used for the data filesystem:
\if\defstring{\labboard}{stm32mp1}
  \begin{verbatim}
  mkpart data 147456s 100%
  \end{verbatim}
\fi
\if\defstring{\labboard}{stm32mp2}
  \begin{verbatim}
  mkpart data 154922s 100%
  \end{verbatim}
\fi
\end{itemize}

Use \code{quit} when you are done.

\section{Data partition on the SD card}

Using the \code{mkfs.ext4} create a journaled file system on the
sixth partition of the SD card:
\ifdefstring{\labboard}{stm32mp1}{
\bashcmd{$ sudo mkfs.ext4 -L data -E nodiscard /dev/mmcblk0p6}}
\ifdefstring{\labboard}{stm32mp2}{
\bashcmd{$ sudo mkfs.ext4 -L data -E nodiscard /dev/mmcblk0p5}}

\begin{itemize}
\item \code{-L} assigns a volume name to the partition
\item \code{-E nodiscard} disables bad block discarding. While this
      should be a useful option for cards with bad blocks, skipping
      this step saves long minutes in SD cards.
\end{itemize}

Now, mount this new partition and move the contents of the
\code{/www/upload/files} directory (in your target root filesystem) into
it. The goal is to use the data partition of the SD card as the storage
for the uploaded images.

Insert the SD card in your board and boot. You should see the
partitions in \code{/proc/partitions}.

Mount this data partition on \code{/www/upload/files}.

Once this works, modify the startup scripts in your root filesystem
to do it automatically at boot time.

Reboot your target system and with the \code{mount} command, check that
\code{/www/upload/files} is now a mount point for the last SD card
partition. Also make sure that you can still upload new images, and
that these images are listed in the web interface.

\section{Adding a tmpfs partition for log files}

For the moment, the upload script was storing its log file in
\code{/www/upload/files/upload.log}. To avoid seeing this log file in
the directory containing uploaded files, let's store it in
\code{/var/log} instead.

Add the \code{/var/log/} directory to your root filesystem and modify
the startup scripts to mount a \code{tmpfs} filesystem on this
directory. You can test your \code{tmpfs} mount command line on the
system before adding it to the startup script, in order to be sure
that it works properly.

Modify the \code{www/cgi-bin/upload.cfg} configuration file to store
the log file in \code{/var/log/upload.log}. You will lose your log
file each time you reboot your system, but that's OK in our
system. That's what \code{tmpfs} is for: temporary data that you don't need
to keep across system reboots.

Reboot your system and check that it works as expected.

\section{Making a SquashFS image}

We are going to store the root filesystem in a SquashFS filesystem in
the fifth partition of the SD card.

In order to create SquashFS images on your host, you need to install
the \code{squashfs-tools} package. Now create a SquashFS image of your
NFS root directory.

Finally, using the \code{dd} command, copy the file system image to
the fifth partition of the SD card.

\section{Booting on the SquashFS partition}

In the U-boot shell, configure the kernel command line to use the
fifth partition of the SD card as the root file system. Also add the
\code{rootwait} boot argument, to wait for the SD card to be properly
initialized before trying to mount the root filesystem. Since the SD
cards are detected asynchronously by the kernel, the kernel might try
to mount the root filesystem too early without \code{rootwait}.

Check that your system still works.

\section{Loading the kernel and DTB from the SD card}

In order to let the kernel boot on the board autonomously, we can
copy the kernel image and DTB in the boot partition we created
previously.

Insert the SD card in your PC, it will get auto-mounted. Copy the
kernel and device tree to the \code{boot} partition.

Insert the SD card back in the board and reset it. You should now be
able to load the DTB and kernel image from the SD card and boot with:
\if\defstring{\labboard}{stm32mp1}
  \begin{ubootinput}
  => load mmc 0:4 %\zimageboardaddr% zImage
  => load mmc 0:4 %\dtbboardaddr% %\dtname%-custom.dtb
  => bootz %\zimageboardaddr% - %\dtbboardaddr%
  \end{ubootinput}
\fi
\if\defstring{\labboard}{stm32mp2}
  => load mmc 0:4 %\zimageboardaddr% Image
  => load mmc 0:4 %\dtbboardaddr% %\dtname%-custom.dtb
  => booti %\zimageboardaddr% - %\dtbboardaddr%
\fi

You are now ready to modify \code{bootcmd} to boot the board
from SD card. But first, save the settings for booting from
\code{tftp}:

\begin{ubootinput}
=> setenv bootcmdtftp %\$%{bootcmd}
\end{ubootinput}

This will be useful to switch back to \code{tftp} booting mode
later in the labs.

Finally, using \code{editenv bootcmd}, adjust \code{bootcmd} so that
the board starts using the kernel from the SD card.

Now, reset the board to check that it boots in the same way from the
SD card.

Now, the whole system (bootloader, kernel and filesystems) is
stored on the SD card. That's very useful for product demos, for
example. You can switch demos by switching SD cards, and the
system depends on nothing else. In particular, no networking is
necessary.
